\documentclass{homework}
\usepackage{homework}

\pgfplotsset{compat=1.16}

\title{Redes Complexas - CPS765 \\[1ex]%%
	6ª Tarefa}
\author{Pedro Maciel Xavier}
\register{116023847}

\begin{document}
	\smaketitle %% review this
	
	Em sua palestra "Your social media "likes" expose more than you think",  Jennifer Golbeck trata de um dos assuntos mais quentes da atualidade. Combustível para uma série de documentários de grandes produtoras, diversos artigos no tema e até mesmo conversas de bar entre alunos de computação e entusiastas, uma análise aprofundada sobre as redes sociais em que emaranhamo-nos é um assunto que, cedo ou tarde, se tornará tão ubíquo quanto estas.\par
	
	Primeiramente, os resultados são sempre impressionantes. Acreditamos guardar muito de nossas personalidades a sete chaves. Da mesma forma, acreditamos conhecer mais de nós mesmos do que nossos pares. Por isso, acaba sendo tão frustrante e até mesmo assustador presenciar situações onde algoritmos são capazes de prever nossos desejos, atividade que pensamos demandar profundo entendimento de nossas personalidades. Para analisar este tipo de fenômeno, Jen traz um olhar bastante instrutivo sobre alguns dos conceitos mais importantes em ciência das redes, dentre eles, a homofilia.\par
	
	Se torna fácil prever uma série de comportamentos quando se pode contar com a hipótese de que as pessoas se agrupam, seja na vida social real ou virtual, conforme as características que compartilham. Por exemplo, pessoas inteligentes tendem a conviver com maior harmonia diante de outras pessoas igualmente inteligentes. Isso revela quão eficaz é a nossa capacidade de reconhecer, entre as milhares de pessoas com quem vamos interagir ao longo da vida, as mais profundas semelhanças.\par
	
	Esta característica marcante de nossa espécie torna o trabalho dos cientistas de dados muito mais fácil. Com pouco material, um algoritmo moderno é capaz de posicionar um usuário em um grupo, cuja semelhança entre seus indivíduos ocorre transcendendo barreiras políticas, geográficas e linguísticas. Muitas das redes em que se encontra imerso revelam ligações entre você e um sem-fim de pessoas que você provavelmente jamais conhecerá.\par
	
	No entanto, nem tudo são flores. Enquanto os algoritmos se aperfeiçoam e o grau de conhecimento acerca de um indivíduo e seus relacionamentos aumenta vertiginosamente, surgem inúmeras questões acerca da privacidade individual, da gestão das informações e das regras de tomadas de decisão. As redes virtuais se tornam cada vez mais redes reais a medida que influenciam na vida cotidiana dirigindo intenções de consumo, processos seletivos de posições de trabalho e até mesmo sugerindo relacionamentos. Por fim, acredito que trabalhos como o de Jeniffer são muito importantes para entendermos melhor o mundo em que já vivemos e as consequências daquilo que "curtimos".\par
\end{document}