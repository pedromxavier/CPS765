\documentclass{homework}
\usepackage{homework}

\pgfplotsset{compat=1.16}

\title{Redes Complexas - CPS765 \\[1ex]%%
4ª Tarefa}
\author{Pedro Maciel Xavier}
\register{116023847}

\begin{document}
    \smaketitle %% review this
    
    Quem controla o mundo? Neste seminário, James Glattfelder trás uma aplicação de ciência das Redes no campo da economia. Como é costumeiro em palestras da área, são apresentadas outras situações onde a estrutura da rede é evidente no funcionamento do sistema sob análise.\par
    
    Quando dá o exemplo do formigueiro, James introduz o que acredito ser um dos fenômenos mais importantes e esclarecedores em sua fala: a emergência. Em linhas gerais o que define o comportamento de um sistema complexo são as relações entre os seus componentes, e não as atitudes individuais de cada um. Ou seja, para compreender este tipo de mecanismo é preciso investigar a estrutura mais do que as partes que o compõem.\par
    
    Disto isso, Glattfelder afirma que, até pouco tempo atrás, análises econômicas sob uma perspectiva de redes ainda não possuíam seu posto definido na literatura. Assim, ele começa a apresentar como se deu e qual é a relevância de seu trabalho no pensamento econômico global.\par
    
    Em uma de suas análises, o palestrante buscou estabelecer métricas que refletissem o "poder" ou o "grau de controle" que uma certa pessoa ou instituição possui, a fim de compreender como se dá a dinâmica intrínseca das relações de influência financeira. São exibidas representações gráficas das redes construídas ao longo do estudo onde se vê os grande detentores de ativos formando um núcleo central na malha econômica, intensamente conectados e responsáveis pela gigantesca maioria da atividade monetária.\par
    
    Desde que o professor mencionou a importância das leis de potência na modelagem estatística em redes eu comecei a me perguntar quanto a relação direta deste fato com a visível concentração de renda que observamos em nossa sociedade. Essa questão é abordada diretamente por James quando apresenta a estrutura que se consolida no centro da rede. É surpreendente como este consegue ser um dos mais antigos problemas sociais que enfrentamos sem sucesso e ao mesmo tempo ainda é capaz de ser revisitado sob a ótica de Redes. A proposta é, de fato, muito interessante. \par
    
    Considerando somente os sistemas socioeconômicos da modernidade, podemos citar diversas propostas de alternativas para o capitalismo industrial hegemônico que, em diferentes graus de implementação, não foram capazes de substituí-lo na prática de maneira duradoura. Uma perspectiva de redes caminha na contramão dos últimos séculos de pesquisa macroeconômica, que muito se empenharam em definir modelos econômicos em larga escala, em esferas muito distantes da vida cotidiana.\par
    
    Retomando o cenário da organização social das formigas, o pesquisador dá ênfase ao dizer que o comportamento do formigueiro como um todo é uma propriedade emergente das relações entre cada formiga, ou seja, é resultado do que acontece nos níveis mais profundos e simples desta sociedade. Trazendo para a sociedade humana em tom de conclusão, afirma que as condições sociais e econômicas são fruto das nossas relações interpessoais em um nível muito mais próximo dos cotidianos do que dos governos. Por fim, conclui que as verdadeiras mudanças que solucionarão os dilemas de nossa sociedade devem emergir da análise de nossas atitudes na vizinhança local a qual pertencemos.\par
    
    
    
    

\end{document}