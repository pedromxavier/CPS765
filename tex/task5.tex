\documentclass{homework}
\usepackage{homework}

\pgfplotsset{compat=1.16}

\title{Redes Complexas - CPS765 \\[1ex]%%
5ª Tarefa}
\author{Pedro Maciel Xavier}
\register{116023847}

\begin{document}
    \smaketitle %% review this
    
    Em sua fala, Manuel Lima aborda o tema de Redes sob uma perspectiva de representação do conhecimento humano. Ele começa introduzindo os modelos de Redes mais simples, isto é, as árvores. Representações em árvores estão por todas as partes da extensa enciclopédia da ciência moderna. Inclusive, certa vez, proferiu um professor da UFRJ a seguinte frase: "Todo algoritmo decente acontece em árvore". De fato, uma afirmação um tanto quanto pretensiosa. Mas, não podemos negar, contar com um fator de complexidade $O(\log n)$ no lugar de um termo linear $O(n)$ é bastante interessante, quando possível.\par
    
    Em seguida, ele trás a representação em Redes, quer dizer, Grafos num contexto geral, como uma nova maneira de representar o conhecimento adquirido e que, carrega em sua estrutura um tanto mais complexa uma miríade de novas possibilidades e relações a serem codificadas. Não acredito que esta seja uma forma tão inovadora quanto Manuel propõe, visto que representações em redes já estavam presentes na antiguidade. Um exemplo deveras corriqueiro mora nos primórdios da astrologia, onde estrelas pertencentes a uma mesma constelação eram agrupadas em componentes conexas conforme a sua proximidade. Acontece que os objetos de estudo dos nossos antepassados não demandavam a investigação de relações tão complexas quanto as de hoje e as frutíferas aplicações de redes não se manifestavam com tanta frequência.\par
    
    Depois disso, o palestrante segue com uma série de visualizações de Redes, apresentando diversas formas de tentar compreender a estrutura a partir de suas diferentes projeções. Após sucessivas figuras, Manuel Lima apresenta o que ele diz ser um movimento artístico que procura se expressar através dos elementos básicos dos grafos: os vértices e as arestas. Foram apresentados exemplos de obras de arte retratando o tema, com intervenções em espaços de exposição assim como pinturas onde se via redes de diversas formas, sob um olhar artístico. Não me agradaram muito as obras mostradas e achei que tratar estas obras como parte de um movimento artístico construído sobre a ciência das redes traz um certo exagero. Neste momento se percebe que o palestrante é aficionado por Redes, de uma maneira não muito saudável. De fato, muitas das representações estão, a meu ver, muito mais próximas daquilo que vemos como saída de uma conjunção entre a biblioteca \texttt{graph\_tool} e \texttt{matplotlib} do que de um retrato sobre o papel das redes em nossa vida.\par
    
    Por fim, não achei a palestra tão interessante a fim de motivar o estudo das redes, apesar de possuir enorme valor de divulgação para o público em geral, principalmente pelo seu caráter introdutório, didático e visualmente rico.\par
    
    

\end{document}