\documentclass{homework}
\usepackage{homework}

\pgfplotsset{compat=1.16}

\title{Redes Complexas - CPS765 \\[1ex]%%
1ª Tarefa}
\author{Pedro Maciel Xavier}
\register{116023847}

\begin{document}
    \smaketitle %% review this

    A palestra "\textit{The hidden influence of social networks}", apresentada por \textit{Nicholas Christakis}, conta a trajetória e as descobertas do locutor, um médico que, por conta de um fenômeno bastante presente no seu dia-a-dia, passou a se interessar por ciência de redes e diz que, graças a isso, agora vê a vida de uma maneira muito diferente. O fenômeno em questão começa a ser apresentado levando em consideração a relevante taxa de pessoas que haviam perdido cônjuges ou entes queridos pouco tempo antes do próprio óbito.\par
    %%
    Também conhecido como \textit{Widowhood effect} (efeito de viuvez), falecimentos em cadeia são amplamente documentados na literatura médica universal, muitas vezes atribuídos ao estado de espírito depressivo que costuma suceder este tipo de acontecimento. O acometimento por esta condição varia conforme o gênero, a etnia, a religião, dentre outros fatores\cite{wilcox:03}\cite{abel:09}. Esta condição é um assunto interessante por si só, do ponto de vista psicológico. No entanto, um evento ainda mais profundo chamou a atenção de \textit{Nicholas}: Um amigo do marido da filha de uma paciente telefonou dizendo que estava muito angustiado com a situação desta última. Cuidar da mãe vinha deixando a filha exausta, e seu marido se encontrava aflito por isso também. Vendo a situação do amigo, portanto, sentiu-se também inquieto com a conjuntura.\par
    %%
    A sucessiva piora de uma paciente se mostrou suficiente para afetar uma pessoa que não se encontrava no círculo familiar e que, mais precisamente, estava a três pessoas de distância desta. A partir desse momento, \textit{Christakis} inicia uma série de investigações acerca dos efeitos que cada indivíduo provoca sobre os seus pares, os pares dos seus pares, e assim por diante. É feito um convite aos espectadores que olhem para os relacionamentos sob uma perspectiva de redes, isto é, que se analise as diversas relações humanas prestando atenção aos padrões que a estrutura inerente proporciona.%%
    \par
	%%
	Após apresentar diversos exemplos, desde os casos de obesidade até o nível de felicidade individual, o apresentador indica como patologias e emoções que, a princípio, não possuem caráter contagioso mas podem ser compreendidas quanto à sua forma de propagação. Após apresentar visualizações dos resultados da sua pesquisa, sugere à plateia um olhar sobre a sociedade como um grande organismo. Esta visão, de fato, já é de grande familiaridade no campo da sociologia. Em sua maior obra, "\textit{A Divisão Social do Trabalho}", \textit{Émile Durkheim} debruça sobre a emergente sociedade industrial apontando para a chamada solidariedade orgânica, que é aquela observada em comunidades que se comportam como organismos vivos, onde cada indivíduo é identificado pela sua função no sistema, ou seja, sua profissão\cite{durkheim:19.97}. Um olhar poético sobre essa questão pode ser encontrado na obra de \textit{Leminsky}, "\textit{Corpo não mente}"\cite{leminsky:19.97}%%
	\par
	%%
	Ao fim de sua fala, \textit{Nicholas Christakis} manifesta ao público a importância de se manter as redes e suas conexões, de uma maneira geral, argumentando que estas são meios de promover o bem. No entanto, outros especialistas alertam para as armadilhas e revezes da Internet, uma das maiores redes que construímos. Em sua palestra "\textit{How we need to remake the internet}", também na plataforma \textit{TED}, \textit{Jaron Lanier} atenta para o modelo de negócio que a estrutura da rede proporciona, e como a construção da mesma é responsável por diversos problemas que afetam profundamente a privacidade, as escolhas e o bem-estar individual de seus componentes\cite{lanier:18}. É interessante contemplar a visão otimista que \textit{Christakis} nutre sobre as redes, e temos que considerar que ele está falando delas num contexto deveras generalista. No entanto, acredito que é necessário dedicar também bastante atenção ao crescente número de redes tóxicas que alimentamos todos os dias. Um fato que se tira de todas as visões aqui apresentadas é que estamos imersos em muito mais redes do que imaginamos, e aprender suas características é fundamental para perpetuar a nossa existência de maneira saudável.%%
	\par
	%%
	 
	
	\thebibliography{10}
	
	\bibitem{wilcox:03} Wilcox, Sara; Evenson, Kelly R.; Aragaki, Aaron; Wassertheil-Smoller, Sylvia; Mouton, Charles P.; Loevinger, Barbara Lee (September 2003). "The effects of widowhood on physical and mental health, health behaviors, and health outcomes: The Women's Health Initiative". Health Psychology. 22 (5): 513–522. 

	\bibitem{abel:09} Abel, Ernest L.; Kruger, Michael L. (2009). "The Widowhood Effect: A Comparison of Jews and Catholics". OMEGA: Journal of Death and Dying. 59 (4): 325–337.
	
	\bibitem{lanier:18} Lanier, Jaron (2018). How we need to remake the internet [Video file]. Retrieved from \texttt{https://www.ted.com/talks/jaron\_lanier\_how\_we\_need\_to\_remake\_the\_internet}
	
	\bibitem{durkheim:19.97} Durkheim, Emile. The Division of Labor in Society. Trans. W. D. Halls, intro. Lewis A. Coser. New York: Free Press, 1997
	
	\bibitem{leminsky:19.97} Leminsky, Paulo. Corpo não mente. Revista Corpo a Corpo, 1987, p. 97-98.

\end{document}