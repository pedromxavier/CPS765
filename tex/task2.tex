\documentclass{homework}
\usepackage{homework}

\pgfplotsset{compat=1.16}

\title{Redes Complexas - CPS765 \\[1ex]%%
2ª Tarefa}
\author{Pedro Maciel Xavier}
\register{116023847}

\begin{document}
    \smaketitle %% review this

    Em "\textit{I am my connectome}"~ ("Eu sou o meu conectomema", tradução livre), o neurocientista \textit{Sebastian Seung} apresenta um questionamento muito antigo cujas tentativas de solução provocaram todo tipo de mudança na sociedade, mesmo não trazendo nenhuma resposta definitiva: Quem ou o que somos nós? O que define o indivíduo? A primeira proposta do palestrante é dizer que somos nossos genes. De fato, sabemos que nossa genética governa muito do que somos, desde as características físicas que nos identificam até o surgimento de doenças. Principalmente nos momentos que seguiram as primeiras descobertas no campo da genética, era comum ter quem acreditasse que os genes determinariam o comportamento e as atitudes de cada um, acarretando numa forma de destino. Esta visão não desapareceu totalmente, apesar de ter se tornado menos popular conforme um maior conhecimento sobre o assunto foi adquirido. No entanto, após a provocação, \textit{Seung} afirma: somos mais do que nossos genes.
    %%
    Neste momento, é introduzido o tópico principal: o conectomema, isto é, o mapa que representa as nossas conexões cerebrais em sua totalidade, onde se conhece em detalhe as ligações de cada neurônio. Ao afirmar que somos nossos respectivos conectomemas, o pesquisador passa a tratar a genética como um leque de possibilidades, que serão ou não exploradas segundo a configuração estampada pelo gigantesco mosaico das sinapses. Com isto, ele traz de volta a importante pergunta, "O que somos nós?", só que dessa vez em sua forma mais recente: "Como é um conectomema?"%%
    \par
    %%
    Já foi dada a definição abstrata da coisa. Contudo, não é uma tarefa simples mapear de um ser humano os bilhões de neurônios, tão pequenos e densamente emaranhados. É ainda necessário que diversas tecnologias sejam desenvolvidas para que se possa realizar esta tarefa homérica. A investigação proposta por \textit{Seung} e seus colegas é, sem dúvidas, uma imersão direta no ponto máximo desta pergunta que acompanha a trajetória humana.%%
    \par
    %%
    Toda sociedade tem seu mito de criação e sua atribuição de propósito, dos pés de \textit{Brahma} ao \textit{Nirvana}, do Éden ao paraíso, da cegonha ao sucesso. Com o florescer do pensamento cartesiano no ocidente, já premeditada pelos pensadores católicos da idade média, a separação entre corpo e mente (alma) se tornaria o \textit{modus ponens} da introspecção em nossa civilização. Os avanços da medicina e da compreensão anatômica dos nossos mecanismos e funções traria então uma resposta aparentemente precisa para as perguntas iniciais, a medida que sabemos onde reside a nossa personalidade, guardamos nossas memórias e organizamos as faculdades cognitivas.%%
    \par
    %%
    No entanto, um primeiro contato com uma visão materialista do ser pode se mostrar bastante desesperador. Correntes de pensamento como o Espiritismo de \textit{Allan Kardec} trariam propostas de conciliação entre as crescentes descobertas do séc. XIX e uma perspectiva de existência que possuísse significado. Em seu livro de maior alcance, fala diversas vezes nos aspectos reconfortantes da doutrina Espírita. Por outro lado, talvez pelos sentimentos que seus pensamentos provocam, levaria ainda muito tempo para que os filósofos niilistas e existencialistas fossem ouvidos em sua proposta que traz a abdicação da necessidade de um sentido para a existência.%%
	\par
	%%
	Mais adiante, o palestrante enuncia de maneira mais objetiva as pretensões e implicações imediatas de sua pesquisa. O objetivo principal é, contudo, mapear extensivamente as conexões do cérebro humano (ou de outros mamíferos, como roedores) a fim de compreender os mecanismos responsáveis pelo seu funcionamento assim como pelas suas disfunções e desordens. De maneira simples, estamos buscando explicar fenômenos como raciocínio, sentimentos e consciência a partir das propriedades emergentes da rede que todos possuímos em nossos respectivos crânios. Não é possível, todavia, caminhar neste assunto sem esbarrar em outras centenas de perguntas. O campo da computabilidade, muito prolífico no século passado, já havia levantado uma questão importante: Uma máquina capaz de raciocinar possui os meios necessários para compreender o seu próprio funcionamento? Sem entrar nesse mérito, \textit{Seung} comenta a possibilidade de que as ambições de projetos como o seu talvez não sejam alcançáveis, por razões outras, mas isso não parece ser uma possibilidade em que ele de fato acredite, e se mostra motivado ao apresentar as técnicas utilizadas no mapeamento atualmente.%%
	\par
	%%
	Por fim, \textit{Sebastian} resume as ideias apresentadas através de uma poderosa metáfora: Um rio que corre sobre seu leito. Diz ele que o terreno orienta os caminhos por onde a água deve correr. Com o tempo, porém, a água passa a esculpir o trajeto e desenha seu percurso de maneiras diversas ao longo de sua existência. Assim, conclui, podemos compreender o papel da genética como um projeto inicial, cuja realização se concretiza em nossas conectomemas, moldadas a partir de experiências, escolhas e reflexões.%%
	\par
	%%
	Penso que a palestra aproxime quem assiste de um futuro iminente onde temos, como sociedade, lições importantes para aprender. As possíveis descobertas dessa área de pesquisa podem tornar mandatória a reflexão sobre temas que anteriormente evitamos ou simplesmente não tivemos as ferramentas necessárias para contemplar. 
	 
\end{document}