\documentclass{homework}
\usepackage{homework}

\pgfplotsset{compat=1.16}

\title{Redes Complexas - CPS765 \\[1ex]%%
3ª Tarefa}
\author{Pedro Maciel Xavier}
\register{116023847}

\begin{document}
    \smaketitle %% review this
    
    Nesta palestra sobre Ciência das Redes, o tema principal é simplesmente o mais antigo problema que, como espécie, buscamos resolver: as doenças. Porquê adoecemos e de que maneiras podemos compreender as enfermidades que nos acometem tantas vezes ao longo da vida? O apresentador, \textit{Albert-László Barabási}, físico Húngaro que hoje pesquisa Redes Complexas e coordena diversas iniciativas no campo, tem mostrado como sua área de pesquisa enxerga as questões fundamentais da medicina.%%
    \par
    %%
    A motivação surge da comparação entre o corpo humano e um automóvel. Já que automóveis são uma tecnologia amplamente conhecida e documentada por nós, é relativamente fácil consertar eventuais defeitos, uma vez que se tenha as peças necessárias. No caso da nossa saúde, isso não se verifica. Com tudo o que conhecemos sobre o funcionamento e composição da nossa fisiologia ainda não somos capazes de solucionar milhares de doenças. Um fato apresentado que ajuda a justificar essa relação assimétrica entre nós e os automóveis é que não conhecemos o corpo humano tão bem assim, principalmente quando ampliamos a resolução com que o observamos. A relação que nossos genes expressam sobre as organelas e proteínas que nos compõem configura ainda uma miríade de mistérios que caminhamos lentamente para desvendar.%%
    \par
    %%
    A perspectiva central de seu trabalho é, portanto, conhecer melhor o funcionamento intrínseco do organismo. Ele exemplifica seu interesse através da visualização de um grafo cujos vértices são proteínas, células e componentes do genoma humano, ligados conforme suas interações no corpo. Dessa maneira, mostra que é possível agrupar estes componentes conforme a expressividade de cada um diante de uma certa patologia.%%
    \par
    %%
    Uma visão em redes permite aos pesquisadores analisar as doenças sob uma outra ótica, que em muito difere da visão médica tradicional. As implicações disso podem ser tão profundas no campo da medicina quanto foi a transformada de \textit{Fourier} no processamento de sinais. Ambas são capazes de lançar novas maneiras de enxergar uma mesma situação, onde correlações entre objetos aparentemente desconexos no diagrama fisiológico podem se mostrar relacionados.%%
	\par
	%%
	\textit{Barabási} reitera que não é médico e que, portanto, não trata diretamente nenhum paciente com o seu trabalho, mas pode auxiliar muitos médicos e seus pacientes encontrando novos tratamentos e explicações que aumentem a compreensão acerca do funcionamento do corpo. De fato, em nenhum momento anterior de nossa história a saúde coletiva dependeu tanto de profissionais de outras áreas. Na maior parte do tempo, os médicos foram integralmente responsáveis pelo estudo e tratamento das enfermidades.%%
	\par
	%%
	A contribuição de físicos, matemáticos, engenheiros e sociólogos, além de gigantescos centros de processamento de dados espalhados pelo mundo, chega em um momento crítico da prática médica em todo mundo, onde observa-se uma frequente mercantilização e precarização do atendimento, principalmente no quesito humano durante as consultas e tratamentos. Um novo olhar sobre a origem e evolução das doenças pode ser um excelente ponto de partida para a construção de um novo olhar sobre a medicina como um todo.%%
	\par
	%%
	Pessoalmente, senti certa força motivadora com essa palestra. Desde que entrei na faculdade, em 2016, até este momento em que estou prestes a concluir o curso (ECI), sempre senti um certo incômodo com as origens e caminhos que a nossa profissão propõe. Isso se acentuou muito durante a pandemia, onde sempre me perguntava sobre a importância do Engenheiro de Computação numa sociedade física e mentalmente doente. A sensação foi muitas vezes de impotência diante desse cenário catastrófico. Mas nem tudo não são flores, e esta disciplina tem sido fonte de motivação nessa reta final do começo da minha formação.

\end{document}